\documentclass[a4paper, 12pt]{report}

\begin{document}

\title{Cochlear Implant Electrode Detection (CIED)}
\author{Tim Ogi, Lucien Hinderling, Dona Lerena}
\maketitle

\section{Introduction}
Nowadays, profound hearing losses are most commonly treated with cochlear implants (CIs), which can restore the hearing by directly stimulating the auditory nerves with electrical impulses. A CI system consists of 1) an external audio processor, 2) the transmission coil, 3) the receiver/stimulator unit and 4) the electrode array, which is inserted into the cochlea. As the cochlea is organized tonotopically, the position of the inserted electrodes is of atmost importance to estimate how a particular frequency is perceived by the patient. Therefore, the aim of this project is to determine the angular depth of the 12 inserted electrodes, using pre-and postoperational computer tomography (CT) images from multiple patients.


\section{Methods}
\subsection{Spectral center}
The spectral center has been approximated as the center of the circle built by the three innermost electrodes.
$$\overrightarrow{M1}+s*\overrightarrow{n1}=\overrightarrow{M2}+r*\overrightarrow{n2}$$ ,where \overrightarrow{M1}=middle point between electrode 1 and 2, \overrightarrow{M2}=middle point between electrode 2 and 3, s/r= some numbers, \overrightarrow{n1}=normal vector to the vector connecting electrode 1 and 2, \overrightarrow{n2}=normal vector to the vector connecting electrode 2 and 3.


\section{Results}

\section{Discussion}

\section{Contributions}

\end{document}